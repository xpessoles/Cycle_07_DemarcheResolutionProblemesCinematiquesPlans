\documentclass[10pt,fleqn]{article} % Default font size and left-justified equations
\usepackage[%
    pdftitle={CIN : Démarche de résolution des problèmes de cinématique graphique},
    pdfauthor={Xavier Pessoles}]{hyperref}
    
\input{style/new_style}
\input{style/macros_SII}

\usepackage{multicol}
\fichetrue
%\fichefalse

\proftrue
%\proffalse

\tdtrue
%\tdfalse

\courstrue
\coursfalse

\def\discipline{Sciences \\Industrielles de \\ l'Ingénieur}
\def\xxtete{Sciences Industrielles de l'Ingénieur}

\def\classe{PTSI}
\def\xxnumpartie{Cycle 7}
\def\xxpartie{Démarche de résolution des problèmes de cinématique graphique\\
Analyser, Résoudre}

\def\xxnumchapitre{Chapitre 1}
\def\xxchapitre{Résolution graphique}

\def\xxtitreexo{Exercices divers}
\def\xxsourceexo{}%\hspace{.2cm} D'après concours Mines-Ponts -- 2013.}


\def\xxposongletx{2}
\def\xxposonglettext{1.45}
\def\xxposonglety{20}
\def\xxonglet{Cycle 7 -- Ch. 1}

\def\xxactivite{Colles}
\def\xxauteur{\textsl{Xavier Pessoles}}

\def\xxcompetences{%
\textsl{%
%\textbf{Savoirs et compétences :}\\
%\noindent \textbf{Résoudre :} à partir des modèles retenus :
%\begin{itemize}[label=\ding{112},font=\color{ocre}] 
%\item choisir une méthode de résolution analytique, graphique, numérique;
%\item mettre en \oe{}uvre une méthode de résolution.
%\end{itemize}
%\begin{itemize}[label=\ding{112},font=\color{ocre}] 
%\item \textit{Rés -- C1.1 :} Loi entrée sortie géométrique et cinématique -- Fermeture géométrique.
%\end{itemize}
%
%\noindent \textit{Mod2 -- C4.1 :} Représentation par schéma bloc.
}}

\def\xxfigures{
%\includegraphics[width=.8\textwidth]{images/prot_01}
}%figues de la page de garde

\def\xxpied{%
Cycle 7 -- Démarche de résolution des problèmes de cinématique \\
Ch. 1 : Résolution graphique -- \xxactivite%
}


\setcounter{secnumdepth}{5}
%---------------------------------------------------------------------------


\begin{document}
%\chapterimage{png/Fond_Cin}
\input{style/new_pagegarde}
\vspace{5cm}
\pagestyle{fancy}
\thispagestyle{plain}


\def\columnseprulecolor{\color{ocre}}
\setlength{\columnseprule}{0.4pt} 

\begin{flushright}

\end{flushright}

 \renewcommand{\baselinestretch}{1.2}
%\setlength{\parskip}{2ex plus 0.5ex minus 0.2ex}

\section{Nivelleur de quai}
\begin{flushright}
\textbf{\textit{D'après ressources de ???}}
\end{flushright} 
\setcounter{exo}{0}
\begin{minipage}[c]{.45\linewidth}
Pour résoudre le problème de la différence de niveau entre un quai 
de chargement et le plancher d'un camion, on utilise des niveleurs de 
quai.

La bavette de liaison 1 permet de faire la liaison entre le plancher du 
camion  et le quai 0. Son mouvement est imposé par un vérin 
hydraulique 4+5.

La tige  4  du vérin rentre dans le corps 5 du vérin  à la vitesse de 
$3,6\;cm/s$.

Échelle des vitesses conseillée : $1\;cm \Longleftrightarrow 2\;cm/s$.

L’échelle du dessin est de 1/10.

\end{minipage}\hfill
\begin{minipage}[c]{.45\linewidth}

\begin{center}
\includegraphics[width=.8\textwidth]{images/fig4_1}
\end{center}
\end{minipage}
\subparagraph{}
\textit{Déterminer la vitesse de rotation du bec de liaison 1 par rapport à la table 0 : $||\vecto{1}{0}||$.}

\newpage
$$
\quad
$$
\vspace{5cm}

\begin{center}
\includegraphics[width=.8\textwidth]{images/fig4_2}
\end{center}


%\section{Machine de poinçonnage}
%\begin{flushright}
%\textbf{\textit{D'après ressources de Florestan Mathurin.}}
%\end{flushright}
%
%On étudie une machine de poinçonnage. Cette machine permet de faire des trous dans les pièces dont la forme le nécessite. Ces trous sont obtenus par arrachage de matière lors de la percussion à haute vitesse d'un outil (appelé poinçon) avec la pièce en question.
%
%\begin{center}
%\includegraphics[width=.9\textwidth]{images/fig1_1}
%
%\includegraphics[width=.7\textwidth]{images/SysML/Exigences}
%\end{center}
%
%\begin{obj}
%L'objectif est de vérifier que l'exigence 1.4.1 est vérifiée.
%\end{obj}
%
%
%Le schéma cinématique de la mise en mouvement du poinçon, dans la machine est fournit sur la figure de la page suivante. Un moteur impose un mouvement de rotation de la pièce 1. Ce mouvement est transformé par les pièces 2, 3 et 4, jusqu'à être changé en mouvement de translation alternative du poinçon 5. 
%
%\subparagraph{}
%\textit{La pièce 1 tourne à 200 tr/min. La distance OA est de 4cm. Déterminer $||\vectv{A}{1}{0}||$.}
%
%\subparagraph{}
%\textit{Le sens de rotation de la pièce 1 est donné sur la figure. Tracer sur cette figure $\vectv{A}{1}{0}$. Échelle graphique : $1m/s = 6cm$.}
%
%\subparagraph{}
%\textit{Déterminer, en argumentant votre réponse, $\vectv{B}{2}{0}$.}
%
%\subparagraph{}
%\textit{Déterminer, en argumentant votre réponse, $\vectv{D}{5}{0}$.}
%
%\subparagraph{}
%\textit{Conclure quant à la capacité de la machine de poinçonnage à satisfaire le critère de vitesse de déplacement du cahier des charges.}
%
%\vspace{6cm}
%
%\begin{center}
%\includegraphics[width=.6\textwidth]{images/fig2}
%\end{center}

\newpage

\section{Benne de camion}
\begin{flushright}
\textbf{\textit{D'après ressources de Florestan Mathurin.}}
\end{flushright}
\setcounter{exo}{0}
On se propose d'étudier le système qui assure l'ouverture d'une benne de camion de ramassage d'ordures.

\begin{minipage}[c]{.3\linewidth}
\begin{center}
\includegraphics[width=.95\textwidth]{images/fig3_1}\hfill
\end{center}
\end{minipage} \hfill
\begin{minipage}[c]{.65\linewidth}
\begin{center}
\includegraphics[width=.95\textwidth]{images/SysML/Exigences_Benne}
\end{center}
\end{minipage}

\begin{obj}
L'objectif est de vérifier que l'exigence 1.4.1 est vérifiée.
\end{obj}


Le schéma cinématique de la mise en mouvement du système est fourni sur la figure suivante. Un vérin impose le mouvement du système. Dans la position donnée, la vitesse de sortie de la tige 2 par rapport au corps du vérin 1 est de $0,1\; m/s$ (échelle des vitesses : 3cm pour 0,1 m/s).


\begin{center}
\includegraphics[width=.6\textwidth]{images/fig4}
\end{center}

\textbf{Les tracés sont à réaliser sur la figure page suivante.}
\subparagraph{}
\textit{Déterminer graphiquement avec les justifications utiles $\vectv{B}{3}{0}$ puis $\vectv{F}{3}{0}$}


\subparagraph{}
\textit{Déterminer $\omega(3/0)$ et conclure vis-à-vis du cahier des charges $(BO=6m)$.}

La benne est munie d'une porte 4 qui s'ouvre lorsque 3 s'incline.


\subparagraph{}
\textit{Déterminer graphiquement avec les justifications utiles $\vectv{C}{5}{0}$ et $\vectv{G}{5}{0}$.}


\subparagraph{}
\textit{Déterminer $\omega_{5/3}$.}


$$ \quad $$
\vspace{6cm}

\begin{center}
\includegraphics[width=.7\textwidth]{images/fig5}
\end{center}



\newpage

\section{Porte d'autobus}
\begin{flushright}
\textbf{\textit{D'après ressources de Florestan Mathurin.}}
\end{flushright}

\setcounter{exo}{0}
On considère un système d'ouverture de porte d'autobus dont on donne un extrait de cahier des charges ci-dessous.

\begin{minipage}[c]{.3\linewidth}
\begin{center}
\includegraphics[width=.95\textwidth]{images/fig6_1}\hfill
\end{center}
\end{minipage} \hfill
\begin{minipage}[c]{.65\linewidth}
\begin{center}
\includegraphics[width=.95\textwidth]{images/SysML/Exigences_Bus}
\end{center}
\end{minipage}
 
La figure de la page suivante représente le schéma du mécanisme actionneur d'une porte (3) d'autobus (en vue dessus). Au dessus de la porte, un vérin pneumatique (air comprimé) (4,5) entraîne une bielle (2) en liaison pivot avec la carrosserie (1). Le bras (AB), encastré à la bielle (2), entraîne le battant de porte (3) qui est guidé par un maneton (C) se déplaçant dans la rainure. L'amplitude de rotation de la bielle (2) de 90 degrés environ permet d'obtenir les positions extrêmes (ouvert/fermé) du battant (3). 

Pour tous les tracés des vitesses on prendra 10mm/s pour 5mm.
 La vitesse de sortie du vérin lors de l'ouverture de la porte d'autobus est $||\vectv{F}{4}{5}||=50mm/s$


\subparagraph{}
\textit{Déterminer graphiquement le vecteur vitesse $\vectv{F}{4}{1}$ en justifiant la démarche suivie. }

\subparagraph{}
\textit{Déterminer, par équiprojectivité, le vecteur vitesse $\vectv{B}{3}{1}$ en justifiant la démarche suivie.}

\subparagraph{}
\textit{Donner la direction du vecteur vitesse $\vectv{C}{3}{1}$. En déduire la position du centre instantané de rotation de la porte (3) par rapport au bâti (1) noté $I_{31}$.}

\subparagraph{}
\textit{Déterminer graphiquement le vecteur vitesse $\vectv{C}{3}{1}$ en justifiant la démarche suivie.}

\subparagraph{}
\textit{Conclure quant à la capacité de  la porte d'autobus à l'exigence 1.4.1.}

\subparagraph{}
\textit{Déterminer le CIR du mouvement de (4) par rapport à 1.}

\newpage

$$\quad$$

\vspace{15cm}

\begin{center}
\includegraphics[width=.8\textwidth]{images/fig7}
\end{center}




\newpage
\section{Commande d'ouverture de soupape}
\begin{flushright}
\textbf{\textit{D'après ressources de Jean-Pierre Pupier.}}
\end{flushright} 
\setcounter{exo}{0}

\begin{minipage}[c]{.5\linewidth}

Le dessin ci-contre représente la commande d'ouverture d'une soupape montée sur une moto HONDA 125 CG.

Un dessin simplifié de cette commande est donné sur le document format A3.

Elle comprend :
\begin{itemize}
\item un bâti \textbf{0} considéré comme fixe;
\item une came \textbf{1} tournant à $250\; rad/s$ autour d'un point fixe $A$;
\item un linguet \textbf{2} ayant un mouvement de rotation autour d'un point fixe $B$;
\item une tige de culbuteur \textbf{3} transmettant le mouvement à la partie haute du cylindre;
\item un culbuteur \textbf{4} destiné à inverser le sens du mouvement. Le culbuteur \textbf{4} tourne autour d'un point fixe $C$;
\item une soupape \textbf{5}.
\end{itemize}

Le dessin est représenté à l'échelle \textbf{1,5 : 1}. On veut calculer, pour la configuration donnée, la vitesse de déplacement de la soupape.

\end{minipage}\hfill
\begin{minipage}[c]{.48\linewidth}
\begin{center}
\includegraphics[width=.95\textwidth]{images/soupape_1}
\end{center}
\end{minipage}

\subparagraph{}
\textit{Calculer la norme $\vectv{I}{1}{0}$ en $mm/s$.}

\subparagraph{}
\textit{Dessiner la sur le document A3 en adoptant l'échelle : $20\; mm/s \leftrightarrow 1 mm$ (c'est très long mais c'est normal).}

\subparagraph{}
\textit{En justifiant vos résultats, trouver graphiquement les vitesses suivantes :\\
\begin{minipage}[c]{.22\linewidth}
\begin{itemize}
\item [$\bullet$] $\vectv{I}{1}{0}$;
\item [$\bullet$] $\vectv{I}{2}{0}$;
\end{itemize}
\end{minipage}\hfill
\begin{minipage}[c]{.22\linewidth}
\begin{itemize}
\item [$\bullet$] $\vectv{D}{2}{0}$;
\item [$\bullet$] $\vectv{D}{3}{0}$;
\end{itemize}
\end{minipage}\hfill
\begin{minipage}[c]{.22\linewidth}
\begin{itemize}
\item [$\bullet$] $\vectv{E}{3}{0}$;
\item [$\bullet$] $\vectv{E}{4}{0}$;
\end{itemize}
\end{minipage}
\hfill
\begin{minipage}[c]{.22\linewidth}
\begin{itemize}
\item [$\bullet$] $\vectv{J}{4}{0}$;
\item [$\bullet$] $\vectv{J}{5}{0}$.
\end{itemize}
\end{minipage}}

\subparagraph{}
\textit{Expliquer en dessinant à main levée un croquis du mécanisme à échelle réduite comment trouver le centre instantané de rotation du mouvement 3/0.}

\subparagraph{}
\textit{Situer approximativement la position de ce CIR.}

\newpage
\begin{center}
\includegraphics[width=\textwidth]{images/soupape_A3}
\end{center}

\end{document}


